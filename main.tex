\documentclass[9pt,twocolumn,twoside]{optica}
\setboolean{shortarticle}{false}
\setboolean{minireview}{false}

\title{Projektowanie geometrii sensorów piezoelektrycznych - elektryczna odpowiedź elemenu piezoelektrycznego na wymuszenie mechaniczne}%Licznik piezoelektryczny

\author[1,*]{Jan Szymenderski}
\author[2,**]{Przemysław Sałapata}

\affil[1]{Politechnika Poznańska, Wydział Elektryczny, pl. Marii Skłodowskiej-Curie 5, 60-965 Poznań}
%\affil[2]{School of Science, University of Technology, 2000 J St. NW, Washington DC, 20036, USA}
%\affil[3]{School of Optics, University of Technology, 2000 J St. NW, Washington DC, 20036, USA}

\affil[*]{e-mail: jan.szymenderski@put.poznan.pl}
\affil[**]{e-mail: przemyslaw.salapata@student.put.poznan.pl}


% To be edited by editor
% \dates{Compiled \today}

%\ociscodes{(140.3490) Lasers, distributed feedback; (060.2420) Fibers, polarization-maintaining; (060.3735) Fiber Bragg gratings.}

% To be edited by editor
% \doi{\url{http://dx.doi.org/10.1364/optica.XX.XXXXXX}}

\begin{abstract}
Do napisania streszczenie artykułu.
%Tutaj streszczenie
\end{abstract}

\setboolean{displaycopyright}{false} %wyśiwetla info o copywrithingu pod abstractem

\begin{document}

\maketitle

\section{Wprowadzenie}

TODO
Do napisania wprowadzenie. Zjawisko piezoelektryczne. Artykuł ma pokazać metodykę projektowania sensorów na konkretnym przykładzie.

\href{http://www.google.com}{przykład linka Google}

\section{Założenia prowadzonych badań}
\label{sec:examples}

%\subsection{Wymagania konstrukcyjne}
Źródłem wymuszeń mechanicznych są owalne ciała (bryły sztywne) o masie $m_s=0.03\div1.10 g$ poruszające się torem ruchu przedstawionym na \ref{fig:route}. Tor wykonany jest ze stalowej rury i w obszarze A (patrz: \ref{fig:route}) następuje sprężysty kontakt ze ścianą toru. Należy nadmienić, że w obszar A uderza 98\% poruszających się ciał. W odrębnych badaniach ustalono również, że prędkość ciała w momencie kontaktu wynosi $v_s=3.0\div7.0$ $\frac{m}{s}$. Wymuszenia mogą pojawiać się minmalnie w odstępach $T_{smin}=TODO$.

\begin{figure}[htbp]
\centering
\fbox{TUTAJ RYSUNEK TORU RUCHU ZIARNA}%\includegraphics[width=\linewidth]{sample}}
\caption{Zakładany tor lotu ciała fizycznego}
\label{fig:route}
\end{figure}

Założono, że miejscem montażu przetwornika jest obszar A na \ref{fig:route}, który stanowi okrąg o średnicy $d_p=TODO mm$. Dodatkowo promień ugięcia płaszczyzny A wynosi $R_A=TODOmm$, a kąt padania ciała na tę powierzchnię $\delta_p=135^{\circ}$. 

Tak ściśle i ciekawie przedstawione założenia stały się dobrym punktem wyjścia do szerzej rozumianych badań. Artykuł na naszkicowanym już przykładzie ukazuje zależność odpowiedzi elektrycznej wybranych przetworników PVDF z energią wymuszenia mechanicznego a przede wszystkim kostrukcją (zwaną także geometrią) układu.

\section{Metodyka badań} %workflow

\section{Stanowisko badawcze}

\section{Selekcja czujnika}
\section{Optymalizacja układu geometrycznego}
%\section{Wyznaczenie charakterystyk}
\section{Podsumowanie}


It is not necessary to place figures and tables at the back of the manuscript. Figures and tables should be sized as they are to appear in the final article. Do not include a separate list of figure captions and table titles.

Figures and Tables should be labeled and referenced in the standard way using the \verb|\label{}| and \verb|\ref{}| commands.

\subsection{Sample Table}

Table \ref{tab:shape-functions} shows an example table.

\begin{table}[htbp]
\centering
\caption{\bf Shape Functions for Quadratic Line Elements}
\begin{tabular}{ccc}
\hline
local node & $\{N\}_m$ & $\{\Phi_i\}_m$ $(i=x,y,z)$ \\
\hline
$m = 1$ & $L_1(2L_1-1)$ & $\Phi_{i1}$ \\
$m = 2$ & $L_2(2L_2-1)$ & $\Phi_{i2}$ \\
$m = 3$ & $L_3=4L_1L_2$ & $\Phi_{i3}$ \\
\hline
\end{tabular}
  \label{tab:shape-functions}
\end{table}

\section{Sample Equation}

Let $X_1, X_2, \ldots, X_n$ be a sequence of independent and identically distributed random variables with $\text{E}[X_i] = \mu$ and $\text{Var}[X_i] = \sigma^2 < \infty$, and let
\begin{equation}
S_n = \frac{X_1 + X_2 + \cdots + X_n}{n}
      = \frac{1}{n}\sum_{i}^{n} X_i
\label{eq:refname1}
\end{equation}
denote their mean. Then as $n$ approaches infinity, the random variables $\sqrt{n}(S_n - \mu)$ converge in distribution to a normal $\mathcal{N}(0, \sigma^2)$.

\section{Sample Algorithm}

Algorithms can be included using the commands as shown in Algorithm \ref{alg:euclid}.

\begin{algorithm}
\caption{Euclid’s algorithm}\label{alg:euclid}
\begin{algorithmic}[1]
\Procedure{Euclid}{$a,b$}\Comment{The g.c.d. of a and b}
\State $r\gets a\bmod b$
\While{$r\not=0$}\Comment{We have the answer if r is 0}
\State $a\gets b$
\State $b\gets r$
\State $r\gets a\bmod b$
\EndWhile\label{euclidendwhile}
\State \textbf{return} $b$\Comment{The gcd is b}
\EndProcedure
\end{algorithmic}
\end{algorithm}

\section*{Funding Information}
National Science Foundation (NSF) (1263236, 0968895, 1102301); The 863 Program (2013AA014402).

\section*{Acknowledgments}

Formal funding declarations should not be included in the acknowledgments but in a Funding Information section as shown above. The acknowledgments may contain information that is not related to funding:

The authors thank H. Haase, C. Wiede, and J. Gabler for technical support.

\section*{Supplemental Documents}
\emph{Optica} authors may include supplemental documents with the primary manuscript. For details, see \href{http://www.opticsinfobase.org/submit/style/supplementary-materials-optica.cfm}{Supplementary Materials in Optica}. To reference the supplementary document, the statement ``See Supplement 1 for supporting content.'' should appear at the bottom of the manuscript (above the references).

%\bigskip \noindent See \href{link}{Supplement 1} for supporting content.

\section*{References}

For references, you may add citations manually or use BibTeX. E.g. \cite{Zhang:14}.

Note that letter submissions to \emph{Optica} use an abbreviated reference style. Citations to journal articles should omit the article title and final page number; this abbreviated reference style is produced automatically when the \texttt{$\setminus$setboolean\{shortarticle\}\{true\}} option is selected in the template, if you are using a .bib file for your references.

However, full references (to aid the editor and reviewers) must be included as well on an informational page that will not count against page length; again this will be produced automatically if you are using a .bib file and have the \texttt{$\setminus$setboolean\{shortarticle\}\{true\}} option selected.



% Bibliography
\bibliography{sample}

% Full bibliography will be added automatically on a new page for Optics Letters submissions. This command is ignored for journal article submissions.
% Note that this extra page will not count against page length.
\bibliographyfullrefs{sample}

%Manual citation list
%\begin{thebibliography}{1}
%\bibitem{Zhang:14}
%Y.~Zhang, S.~Qiao, L.~Sun, Q.~W. Shi, W.~Huang, %L.~Li, and Z.~Yang,
 % \enquote{Photoinduced active terahertz metamaterials with nanostructured
  %vanadium dioxide film deposited by sol-gel method,} Opt. Express \textbf{22},
  %11070--11078 (2014).
%\end{thebibliography}

\end{document} 